
Previous experience of group members on implementing object tracking in video sequences sparked the idea of incorporating a method for evaluating object trackers into the system evaluation. The main reason for this was that an implementation in C++ using OpenCV was already available from a previous computer vision project. The evaluation metric available was the MOTA/MOTP system defined by Bernardin \& Stiefelhagen in (2008) \cite{MOTA}. \\

Initially the evaluation system promised good results as the implementation easily fit into the project architecture. The first evaluation was performed on short RGB test data files. Even though the implementation went smoothly, obtaining ground truth data was a whole other matter. No suitable (top-down view of people passing a doorway) clips with pre-created ground truth data were found, which meant that ground truth data would have to be labeled manually. The ground truth data was on the format specified by the CAVIAR project \cite{CAVIAR}. A program for labeling sequences frame by frame is available on the CAVIAR homepage. Because the tracker evaluation depends on accurate object positions it soon became clear that labeling enough ground truth data by stepping through sequences frame by frame to provide a feasible basis for evaluation would be far to time consuming. This is especially noticeable when one considers the fact that the actual goal of the project is to simply count people passing by, and not to track them with perfect accuracy.

The tracking evaluator is included with the final software because it works well and might be useful if the software is to be used for an other application, for example people tracking outdoors or other open spaces.
