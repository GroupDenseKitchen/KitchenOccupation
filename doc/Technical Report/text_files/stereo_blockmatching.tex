
There are different ways in which one can obtain depth information. The easiest is perhaps to use a Kinect sensor which does all the job for you. However, this sensor is not compatible with the initially intended platform, namely the Raspberry Pi. An alternative to the Kinect is to use stereo vision. The semi-global stereo block matching algorithm proposed in \cite{StereoBM} was tested with partly successful results. The problem is its speed. Each frame takes approximately 300 ms to process which is too much for this application. This could be remedied to some extend by sacrificing image quality. The computation time can be reduced to 50 ms but the result is harder to use in later process steps. This algorithm could have been greatly sped up by a GPU implementation, but this approach was abandoned in favor for the Kinect which is both faster and easier to use. The results can be seen in figure \ref{fig:Stereo} below.

\vspace{1cm}
\begin{figure}[htb]
	\includegraphics[width=\linewidth]{images/stereoComp.png}
	\caption[An example of depth images calculated from a stereo sequence]{\textit{Evaluation of stereo block matching for depth calculations.\\ 
	Top: Left and right image of stereo sequence.\\ 
	Bottom left: Faster approach with lower quality.\\ 
	Bottom right: Slower approach with good quality.}}
	\label{fig:Stereo}  %Skapar referens till figuren
\end{figure}
\newpage

