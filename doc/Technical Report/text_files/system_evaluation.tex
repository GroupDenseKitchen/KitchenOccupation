One of the most important parts of a project is the ability to evaluate it. Without it there is no way to compare results with others... more to come

\subsection{Ground Truth}
TODO: info on acquisition of ground truth:
The evaluation needs access to some sort of ground truth which is defined as the best possible achievable tracking output. Throughout this part of the document an object is defined as a position by the ground truth and a hypothesis as the output from the tracker. The method only allows one-to-one correspondence between objects and hypothesis and in case of conflict the combination yielding the lowest total distance error is chosen. The ground truth data and video clips used for evaluation are all from the CAVIAR project \cite{CAVIAR}.

\subsection{MOTA}
Info about tracker evaluation
MOTA is a measure of accuracy with respect to how many mistakes are made by the tracker. It consists of four variables: misses, false positive, mismatches and number of objects. A miss is when no hypothesis suggested is close enough to an object, close enough being defined by a threshold, T. This is the only design parameter in this method. If the distance between an object and the closest hypothesis is larger than T, the object is considered a miss. If a hypothesis has no object within threshold distance it results it is labeled a false positive. One important feature of a tracker is the ability to keep objects identities correct. If this is not the case and an object is changing identity between frames one mismatch error is added for every change. The number of objects variable is defined as the total number of objects trackable according to the ground truth in the current frame. For a given image sequence, equation \eqref{eq:MOTA} can be used to compute the MOTA.

\subsection{People count}
Info about current evaluation metrics.