Linköping University has several student kitchens all over its campuses where students are provided with the ability to warm food. Critics claim that there are too few student kitchens and that the existing ones are usually overcrowded. That all kitchens are overcrowded at the same time has not been confirmed by sample inspections. One standing hypothesis is that students do not know where all the kitchens are, nor that they want to risk going to a kitchen in another building in case it is full as well.\\

The aim of this project is to develop a system that will provide the students with information regarding student kitchen usage. The system uses an image based approach, estimating the number of people using the kitchens.\\

This document, together with the Code Reference Manual and the User Manual constitutes the final documentation of the project and describes the current implementation together with some failed attempts and conclusions drawn from these. Performance results of the current implementation are presented in section \ref{sec:evaluation} of this document.
