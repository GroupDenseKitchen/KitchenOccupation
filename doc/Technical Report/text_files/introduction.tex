There exists many places in society where the degree of human occupancy and movement flow is desirable to know as basis for decision making. Such data answers if it is necessary to build more rooms and provides knowledge of actual user or consumer patterns. Example usages are measuring resource usage of public spaces, or which part of a store that attracts most people. It provides vast opportunities in resource management, marketing, sales and scheduling. There exist some plausible solutions to estimating the number of people at a location such as using cell phones or motion detectors, but this project aims at an image based approach with the possible benefits of being both cheaper and more robust.

\subsection{Background}
Today Linköping University has many places with similar functionality, e.g. student kitchens where students are provided with the ability to warm food brought with them. Linköping University has several such kitchens all over its campuses. Critics claim that there are too few student kitchens with microwave ovens and that the existing ones usually are overcrowded. That all kitchens are overcrowded at the same time has not been confirmed by sample inspections. One standing hypothesis is that students don't know where all the kitchens are nor that they want to risk going to a kitchen in another building in case that is full as well.\\
\\
Linköping University has an ongoing project with the purpose of enabling the students to see the usage of some of the schools resources (e.g. group rooms) online. The aim of this project is to supply that system with data regarding the usage of student kitchens. It will provide all students with the ability of visualizing the crowdedness of each kitchen, thus providing them with the means of finding the closest, least occupied kitchen available.

\subsection{About this document}
This docment, together with the code reference manual (appendix \textbf{??}) constitutes the final documentation of the project and describes the current implementation together with some failed attempts and conclusions drawn from these. Performance results of the current implementation are presented in section \ref{sec:evaluation} of this document.
