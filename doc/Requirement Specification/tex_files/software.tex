The software part of the system performs all the image processing and analysis needed to detect usage of a room. It will also be cablable of predicting future room occupancy degree based on historical usage. 

A website for the project will be published using the wordpress CMS, it will describe the project and its participants.
 
\subsection{External dependencies}
The software is written using the OpenCV library to handle the image processing, and when possible interfacing with cameras. 
Any visualization and debugging tools are written using the Qt framework. 
OpenNI is used for accessing the Kinect camera. 
An HTTP library is used to handle communication with LIUs REST API.
Cmake and any C++11 capable compiler can be used to compile the source code to a binary.

The website will be hosted on a standard LAMP stack.

\subsection{Compatibility}
The software is possible to build for most major plattforms (Windows/OS X). A REST API is used to communicate the results.

\subsection{Limitations}
For integrity reasons no personally identifiable information can be stored or reported by the system. The processing is done locally for each camera on a Raspberry PI or on a remote computer connected via network or USB. Depending on which of these hardwares the coputational power will differ and therefore limiting the use of computional heavy algorithms.  

\subsection{Software Requirements}
\label{sec:software_req}
\reqtable
{	
	\addreq{The system runs on Windows based plattforms}{1}
	\addreq{The system runs on OS X}{1}
	\addreq{\textbf{Added 2013-11-27:} The system runs on Raspberry PI}{2}
	\addreq{The system is modular with respect to the camera manufacturer and/or network API.
	
	\textbf{Revised 2013-11-27:} The system is modular with respect to any OpenCV-compatible camera and network API}{1}
	\addreq{The system must not store image data}{1}
}