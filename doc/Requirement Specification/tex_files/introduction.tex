There exist many places in society where the degree of human occupancy and movement flow is desirable to know as basis for decision making. Examples of such situations are weather there are enough of a certain type of room, which part of a store that attracts most people or even the variation of flow of people through doors or entrées. Such data answers e.g. if it is necessary to build more rooms and helps understanding user or consumer patterns. It provides vast opportunities in resource management, marketing, sales and scheduling. There exist some plausible solutions to estimating the number of people at a location such as using cell phones or motion detectors, but this project aims at a image based approach with the possible benefits of being both cheaper and more robust.

\subsection{Background}
Today Linköping University have many places with similar functionality and among them are for example student kitchens, where students are provided with the ability to warm food brought with them. Linköping University have several such kitchens all over its campuses. Critics claim that there are too few student kitchens with microwave ovens and that the existing ones usually are overcrowded. That all kitchens are overcrowded at the same time have not been confirmed by sample inspections and one standing hypothesis is that students don't know where all the kitchens are nor that they want to risk going to a kitchen in another building in case that is full as well.\\
\\
The aim is that the result of this project will be used to provide all students with the means of visualising the crowdedness of each kitchen, providing them with the means of finding the closest, least occupied kitchen available.

\subsection{Involved Parties}
Three parties are involved:
\begin{itemize}
\item Liu IT, the Division for IT servidces at Linköping University.
\item Computer Vision Laboratory, Department of Electrical Engineering, Linköping University.
\item A group av students taking the course TSBB11 2013, listed in the \textit{Participants} table, page (ii).
\end{itemize}

\subsubsection{Customer}
Liu IT, represented by Joakim Nejdeby, CIO at Linköping University.

\subsubsection{Supervisor}
Ph.D Fahad Khan at the Computer Vision Laboratory, Department of Electrical Engineering, Linköping University.

\subsection{About this document}
This document contain the requirements of the project. It is divided into different modules or aspects, each with further subdivisions, all containing explanatory text and functional requirements. Each functional requirement is placed in a table of the form showed below.

\reqtable
{
	
}
\subsubsection{Requirement priorities}
Each requirement has three different priority levels (Type), the meaning of each one is presented below:

\begin{enumerate}
	\item Priority level one constitutes a mandatory requirement, meaning this feature has to be fulfilled by at the time specified ion the description. If no time is specified, the requirement has to be fulfilled by the time of the final delivery (see section \ref{sec:delivery}). %Länk till sektionen med leveranskraven
	\item A requirement with priority level two is a requirement to be met if extra time is available.
	\item A level three requirement is more of a suggestion on how to improve the system even further after the final delivery.
\end{enumerate}


\subsection{Definitions}
Text here.

