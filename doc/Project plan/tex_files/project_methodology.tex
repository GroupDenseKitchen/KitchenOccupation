
This section summarizes our version of the SCRUM methodology presented in the SCRUM guide \cite{scrumGuide}, and describes changes we have made to the standard SCRUM. We use two week sprints where each team member is expected to spend 25 percent of their work week on the project.   

\subsection{Documents}
\emph{The Product Backlog (PB)} contain desirable product features which are prioritized. Each backlog item in general corresponds to one or several connected requirements from the requirement specification.
A new Product Backlog item can at any time be added to the Product Backlog and priorities can change at any time. \\
\\
\emph{The Sprint Backlog} (SB) contain a To-Do list made from splitting up one or several Product Backlog items into smaller pieces. These pieces should be things specific enough for implementation. \\
\\
\emph{The Sprint Log Document} (SLD) contains
\begin{itemize}
  \item A Snapshot of the initial SB from the Sprint Planning.
  \item A snapshot of the final SB from the Sprint Review.
  \item A sprint review as well as how much time spent.
\end{itemize} 

\subsection{Definition of responsibilities}
Mattias Tiger is the \emph{Project Manager}, which is the role of the Product Owner and the Scrum Master.\\
Nikolaus West is \emph{responsible for testing}.\\
Martin Svensson is \emph{responsible for documentation}.\\
Gustav Häger is \emph{responsible for system integration}. 

All group members does not have specific roles. According to scrum, group members should not have specific roles, apart from a Scrum leader and a Product owner.  In this case this means three group members are missing a specific role. The other three  roles are specified because of their importance in combination with the fact that this project is done during a very short time, during limited hours.  Documentation, testing and system integration is vital for the project’s progression and therefore group members are given these three roles to ensure their prosecution.  

\newpage
\subsection{Meetings}
At the start of each sprint there is a \emph{Sprint Planning} meeting. During this meeting it is decided what or which items from the PB that will be worked on during the sprint. A new SLD is created and an initial SL snapshot is added.\\
\\
During a sprint there are \emph{biweekly scrum} meetings, which replace the \emph{daily scrum} meetings of regular SCRUM. They are short meetings, around 15 minutes, every Monday and Wednesday. Questions asked and answered during this meeting are:
\begin{itemize}
  \item What have you done since last biweekly scrum?
  \item What will you do before the next biweekly scrum?
  \item What obstacles are impeding your work?
  \item Are there any bugs discovered?
\end{itemize}
\vspace{4mm}
At the end of each sprint there is a \emph{Sprint Review} meeting were the following questions are answered:
\begin{itemize}
  \item What did we finish? 
  \item What did we try?
  \item What solutions did we end up using?
  \item What did we do well?
  \item How can we work better?
\end{itemize}
The \emph{Sprint Review} meeting is a combination of a \emph{Sprint Review} and a \emph{Sprint Retrospective} in the terminology of regular SCRUM. The review is documented and added to the SLD together with the current (final) SL snapshot. SLD is then handed in to the supervisor and the product may be demonstrated for the customer/supervisor if desired.
