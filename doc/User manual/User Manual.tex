\documentclass[10pt, a4paper, twoside]{article}

% Set up the standard margins for the document
% 42.2 left & 15.5 right is same as Forsling, Neymark
% 21.3 top  & 20 bottom is same as Olofsson
\usepackage[left=19.5mm, right=19.5mm, top=21.3mm, bottom=20mm]{geometry}

% Input file character encoding (kinda useless if we don't
% use åäö and stuff, but it doesn't hurt to have it)
\usepackage[utf8]{inputenc}
%\usepackage[swedish]{babel}

% Block Comments
\usepackage{comment}

\usepackage{float}

% No indentation in new paragraph
\usepackage{parskip}

% To include graphics
\usepackage{graphicx}

% More mathematical symbols and fonts
\usepackage{amsmath}
\usepackage{amsfonts}
\usepackage{amssymb}

% Simple list
\usepackage[ampersand]{easylist}
\ListProperties(Hide=100, Hang=true, Progressive=5ex, Style*=$\bullet$ ,
Style2*=-- ,Style3*=$\circ$ ,Style4*=\tiny$\blacksquare$ )


% Clickable internal links
\usepackage{hyperref}
\usepackage[all]{hypcap} % without this the link takes you to the caption, not the top of the image
\hypersetup{ % Settings for links in documnet
	setpagesize = false, % Don't allow hyperref to change page size. Tips från Micke Olofsson
	colorlinks = true,   % No boxes around links
	linkcolor = black,citecolor = black,filecolor = black,urlcolor = black, % don't color links
}

% To include to first page pdf file
\usepackage{pdfpages}

% Add section number to equation and figure number (ex: 5.11 instead of simply 11)
\numberwithin{equation}{subsection}
\numberwithin{figure}{section}
\numberwithin{table}{section}

% Show program code listings in document
\usepackage{listings}

%
% Header stuff
%
\usepackage{fancyhdr}
\setlength{\headheight}{15pt}

\fancyhf{}
\fancyhead[LE, RO]{\thepage}
\fancyhead[RE]{TSBB15 2013: User Manual}
\fancyhead[LO]{Kitchen Occupation}

\fancypagestyle{plain}{ %
\fancyhf{} % remove everything
\renewcommand{\headrulewidth}{0pt} % remove lines as well
\renewcommand{\footrulewidth}{0pt}}
%
% End header stuff
%




\begin{document}

% First page
\includepdf{Cover/cover.pdf}


% Project identity page
\newpage
\pagestyle{fancy}
\pagenumbering{roman}
\setcounter{page}{2} % sets the current page number to 2 

\begin{center}
    \vspace*{4\baselineskip}

	\textbf{\huge Project Kitchen Occupation} \\
	\vspace*{0.5\baselineskip}
	Bilder och Grafik CDIO, HT 2013 \\
	Department of Electrical Engineering (ISY), Link\"{o}ping University
	
	\vspace*{2\baselineskip}
	\textbf{\LARGE Participants}


	{\footnotesize 
	\begin{tabular}{|p{2.7cm}|p{1cm}|p{5cm}|p{2cm}|p{3.4cm}|}
		\hline
		\textbf{Name} & \textbf{Tag} & \textbf{Responsibilities} & \textbf{Phone} & \textbf{E-mail} \\
		\hline
		Mattias Tiger & MT & Project manager & 073--695\,71\,53 & matti166@student.liu.se \\
		\hline
		Erik Fall & EF & -- & 076--186\,98\,84 & erifa226@student.liu.se \\
		\hline
		Gustav Häger & GH & System integration & 070--649\,03\,97 & gusha124@student.liu.se \\
		\hline
		Malin Rudin & MR & -- & 073--800\,35\,77 & malru103@student.liu.se \\
		\hline
		Alexander Sjöholm & AS & -- & 076--225\,11\,74 & alesj050@student.liu.se \\
		\hline
		Martin Svensson & MS & Documentation & 070--289\,01\,49 & marsv106@student.liu.se \\
		\hline
		Nikolaus West & NW & Testing & 073--698\,92\,60 & nikwe491@student.liu.se \\
		\hline
	\end{tabular}
	}

{\footnotesize 
\vspace{0.5\baselineskip}
\textbf{Homepage}: TBA \\
\vspace{1\baselineskip}

\textbf{Customer}: Joakim Nejdeby, Link\"{o}ping University, Origo 3154 \\
\textbf{Customer contact}: 013--28\,17\,57, joakim.nejdeby@liu.se \\
\textbf{Project supervisor}: Fahad Khan, Link\"{o}ping University, fahad.khan@liu.se \\
\textbf{Examiner}: Michael Felsberg, michael.felsberg@liu.se \\
}

\end{center}



% table of contents
\newpage
\tableofcontents
\listoffigures
\listoftables

% list of figures
%\newpage
%\listoffigures


% Document history page
\newpage
\vspace*{5\baselineskip}

\begin{center}
\textbf{\LARGE Document history}

{ \footnotesize 
\begin{tabular}{|p{1cm}|p{2.0cm}|p{5cm}|p{1.5cm}|p{2cm}|}
	\hline

	\textbf{Version} & \textbf{Date} & \textbf{Changes} & \textbf{Sign} & \textbf{Reviewed} \\
	
	\hline
	0.1 & 2013--12--13 & Initial draft & MS & MT\\
	
	\hline
	 &  &  &  &  \\
	
	\hline
\end{tabular}
}
\end{center}


% Blank page
%\newpage
%\thispagestyle{empty}
%\mbox{}



%
% Content start
%
\newpage
\pagenumbering{arabic}

\newpage
\section{Installing the system}
\label{sec:installation}
\subsubsection{Hardware}
Each Kinect camera must be installed above a door with no overlapping view shared with any other Kinect camera. The Kinect must point down or slightly angled towards the room. A minimum distance of 50 cm is required from the lense of the Kinect and the top of the door. Each Kinect must be connected to a power source, and to a device running the system software using USB.
\subsubsection{Software}
There are two versions of the software, one with a calibration and configuration GUI and one lightweight version without a GUI. In order for the lightweight version to work a configuration file, persumably generated by the GUI version, is required.
The configuration file is best generated using the configuration program, and then copied to the system running the non-GUI variant.
\\
\\
Linux, OS-X or Windows is required on the machine running the software. Atleast one Kinect camera must be connected before starting the software. More than one Kinect camear is currently only working on Linux and OS-X. 
Some software libraires are required to compile the program, these are listed in table \ref{reqSoftware}

\begin{tabular}{ l l }
  OpenCV2 & Needed for general image processing  \\
  libFreenect & Needed for communication with kinect on unix like systems  \\
  OpenNI & Needed for communication with kinect on windows systems  \\
  libCurl & Needed to send http requests to the report API \\
  QT5 & Needed for the gui code, not used in headless variant \\
\end{tabular}
\newpage

\section{Setting up the system}
The easiest way to setup the system is by using the GUI. Here the main settings can be adjusted. However, there are some advanced settings that can only be adjusted in the configuration file. More on this later. 

\subsection{Calibration}
\label{sec:setup}
A threshold level is used to adjust the system for the current installation height of the camera. It sets a configuration parameter called lowestDistanceOverFloor. This is the limit of how short a person can be. The threshold should be set so that a "normal" person’s chest is not removed by the thresholding.

\begin{figure}[htb]
	\centering
	\includegraphics[width=\linewidth]{images/Calibration.png}
	\caption[Overview of the entire system]{\textit{Calibrating the lowestDistanceOverFloor threshold. A histogram i shown to help the user to se how much off different heights are present in the image. The selected heights are gray shaded.}}
	\label{fig:lowestDistanceOverFloor_calibration}  %Skapar referens till figuren
\end{figure}

\newpage
\subsection{Configuration}
\label{sec:configuration}
Available configuration settings is checkpoint circles, door mask area, exclusion mask and grayscale height threshold settings. 

The circles should be placed so persons walking into the room inevitable will pass all three lines. They should also be more inside the room compared to the door mask area. A good placement is illustrated in figure \ref{fig:circlePlacement}. Note that the red, most inner circle, includes the upper corners of the door frame. Too small inner circle will cause people to miss it and therefore not detected. 

The door mask should cover the area close to the door where people appear. It is important to make this area big enough, rather too big than too small. It can, but should not cover the upper, most distant, part of the red circle, figure \ref{fig:doorMask} illustrates this.

Exclusion masks should cover areas where people can not walk or appear. This could be areas like tables or areas behind the door (walls in this case), figure \ref{fig:exMask} illustrates this. Note that for long usage of the system, movable furniture should not be excluded. 

This threshold level is used to adjust the system for the current installation height of the camera. It sets a configuration parameter called lowestDistanceOverFloor: This is the limit of how short a person can be. The threshold should be set so that a normal person’s chest height is not removed by the thresholding. ALEX(ELLER NGN ANNAN) FIXA FIGUR HÄR OCH FÖRKLARA MER KANSKE. KANSKE  INTE SKA STÅ HÄR ENS, KANSKE UNDER CALIBRATE THE SYSTEM?


\begin{figure}[H]
	\centering
	\includegraphics[width=\linewidth]{images/Manual2.png}
	\caption[Circle placment]{\textit{A prefered placement of the circles. }}
	\label{fig:circlePlacement}  %Skapar referens till figuren
\end{figure}



\begin{figure}[H]
	\centering
	\includegraphics[width=\linewidth]{images/Manual3.png}
	\caption[Exclusion mask]{\textit{The prefered placement of the door mask, the door mask is the green area.}}
	\label{fig:doorMask}  %Skapar referens till figuren
\end{figure}

\begin{figure}[H]
	\centering
	\includegraphics[width=\linewidth]{images/Manual1.png}
	\caption[Exclusion mask]{\textit{Exclusion mask is marked as red. It covers areas where people can not walk or appear.}}
	\label{fig:exMask}  %Skapar referens till figuren
\end{figure}
\newpage

\subsection{Debugging from GUI}
\label{sec:configuration}

When setting up and especially when developing the system it is convenient to use the debug GUI seen in figure \ref{fig:debug GUI}. From here a lot of information can be obtained. 

In the upper left corner all cameras and their process steps can be seen. These can be selected and then poped to the grid, figure \ref{fig:grid}, by clicking \textit{Pop Window}. One a window configuration in the grid is set it can be saved by clicking \textit{Save grid configuration} under the Grid tab in the menu bar. 

In the upper right corner one can find profiling information for every step in the pipe line. In bottom of the window is the system log. This displays messages from inside the system. 

\begin{figure}[htb]
	\centering
	\includegraphics[width=\linewidth]{images/PosterDebugger.png}
	\caption[The debug GUI]
	{\textit{The debug GUI can be used by advanced users to get more detailed information about the system such as profiling and images of all the inherent process steps of the computer vision pipeline.}}
	\label{fig:debug GUI}  %Skapar referens till figuren
\end{figure}

\begin{figure}[htb]
	\centering
	\includegraphics[width=\linewidth]{images/TheGrid.png}
	\caption[The grid]
	{\textit{The grid like a workspace to which you can send the process steps on which you are currently working. These can be stored to file to minimize manual window handling during development.}}
	\label{fig:grid}  %Skapar referens till figuren
\end{figure}

\newpage


% Force a blank page so the bibliography starts on a new page.
% Comment out if not necessary
%\newpage
%\thispagestyle{fancy}
%\mbox{}
%\begin{thebibliography}{9}
\addcontentsline{toc}{section}{References} % Add an entry for this in the table of contents

\bibitem{scrumGuide}
    Schwaber K., Sutherland J.\\
   \emph{The Scrum Guide: The Definitive Guide to Scrum: The Rules of the Game}\\
     \verb+https://www.scrum.org/Scrum-Guides+,\\
    Jul. 2013\\
    Accessed on September 20th 2013.

\end{thebibliography}



\end{document}
