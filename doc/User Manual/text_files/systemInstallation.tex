\subsection{Hardware}
Each Kinect camera must be installed above a door with no overlapping view shared with any other Kinect camera. The Kinect must point down or slightly angled towards the room. For optimal results the Kinect should be placed approximately 40 cm above the door to be able to detect tall even the tallest persons. Each Kinect must be connected to a power source, and to a device running the system software using USB.

\subsection{Software}
There are two versions of the software, one with a calibration and configuration GUI and one lightweight version without GUI. In order for the lightweight version to work a configuration file, presumably generated by the GUI version, is required. The configuration file is best generated using the configuration program, and then copied to the system running the non-GUI variant. \\

Linux, OS-X or Windows is required on the machine running the software. At least one Kinect camera must be connected before starting the software. More than one Kinect camear is currently only working on Linux and OS-X. Some software libraires are required to compile the program, these are listed in table \ref{reqSoftware}. \\

\begin{table} [hbt]
\begin{center}
  \begin{tabular}{ | l | l | }
    \hline
    \textbf{Software} & \textbf{Comments} \\ \hline
    OpenCV2 & Needed for general image processing  \\ \hline
    libFreenect & Needed for communication with kinect on unix like systems  \\ \hline
    OpenNI & Needed for communication with kinect on windows systems  \\ \hline
    libCurl & Needed to send http requests to the report API \\ \hline
    QT5 & Needed for the gui code, not used in headless variant \\ \hline
  \end{tabular}
  \label{reqSoftware}
  \caption{Software libraries requried.}
\end{center}
\end{table}