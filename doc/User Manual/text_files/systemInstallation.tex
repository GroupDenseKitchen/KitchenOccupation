\subsection{Hardware}
Each Kinect camera must be installed above a door with no overlapping view shared with any other Kinect camera. The Kinect must point down or slightly angled towards the room. For optimal results the Kinect should be placed approximately 40 cm above the door to be able to detect even the tallest persons. Each Kinect must be connected to a power source, and to a device running the system software using USB.

\subsection{Software}
There are two versions of the software, one with a calibration and configuration GUI and one lightweight version without GUI. In order for the lightweight version to work two configuration files need to be located next to the executable. The files are:

\begin{itemize}
\item mainConfig.yml
\item masks.yml
\end{itemize}

When running the GUI version there is exists an additional configuration file specifically for the GUI:

\begin{itemize}
\item guiConfig.yml
\end{itemize}

Default versions of these files are located in the conf folder. When the GUI version is run these files are generated and stored next to the executable. If desired these can be copy to the conf folder and act as defaults in the future.

Linux, OS X or Windows is required on the machine running the software. At least one Kinect camera must be connected before starting the program. Some software libraries are required to compile the program, these are listed in table \ref{reqSoftware} below. \\

\begin{table} [hbt]
\begin{center}
  \begin{tabular}{ | l | l | }
    \hline
    \textbf{Software} & \textbf{Comments} \\ \hline
    OpenCV2 & Needed for general image processing  \\ \hline
    libFreenect & Needed for communication with Kinect on Linux and OS X systems  \\ \hline
    OpenNI & Needed for communication with Kinect on Windows systems  \\ \hline
    libCurl & Needed to send HTTP requests to the report API \\ \hline
    QT5 & Needed for the GUI code, not used in headless variant \\ \hline
  \end{tabular}
  \label{reqSoftware}
  \caption{Required software libraries.}
\end{center}
\end{table}